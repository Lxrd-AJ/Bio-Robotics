\documentclass[journal]{IEEEtran}

% *** CITATION PACKAGES ***
\usepackage[backend=biber,style=ieee]{biblatex} %bibliography with IEEE style.
\addbibresource{ref.bib} %Imports bibliography file
\bibliography{}

% *** GRAPHICS RELATED PACKAGES ***
\ifCLASSINFOpdf
  \usepackage[pdftex]{graphicx}
\else
\fi

% *** MATH PACKAGES ***
\usepackage{amsmath}
\usepackage{amssymb}
\usepackage{gensymb}
\usepackage{hyperref}

\begin{document}
%
% paper title
\title{ELEC6212: Wireless Sunflower Network}
\author{\IEEEauthorblockN{Benjamin~Dix-Matthews\IEEEauthorrefmark{1},
Georgios Psimenos\IEEEauthorrefmark{1},
Ganiyu AJ Ibraheem\IEEEauthorrefmark{1} and 
Aswin K Ramasubramanian\IEEEauthorrefmark{1}}
\IEEEauthorblockA{\IEEEauthorrefmark{1}University~of~Southampton}}

% The paper headers
\markboth{ELEC6212: Biologically Inspired Robotics}%
{University of Southampton}

% make the title area
\maketitle

% As a general rule, do not put math, special symbols or citations
% in the abstract or keywords.
\begin{abstract}
The idea of this project was to create a device capable of ultra-low power sensing and ultra-low power transmission that is capable of harvesting the energy required for its operation from solar sources. The energy harvesting nature of the device would eliminate the significant operational expenditure required to replace batteries in periphery nodes after they have gone flat. As batteries are typically known for having relatively short operational lifetimes (<10 years), our design would instead use a super capacitor in order to store the energy required for operation. This type of “Zero-Energy Sensing” is a new idea with huge potential in the IoT field.
\end{abstract}

\section{Introduction}
% The very first letter is a 2 line initial drop letter followed
% by the rest of the first word in caps.
\IEEEPARstart{T}{his} product would be aimed toward outdoor sensing applications where typical wireless sensing networks could be useful. A specific example of its usage would be a soil moisture detector for monitoring crops. The idea of the Wireless Sunflower Network would be to design a cheap product that could be easily set up by someone unfamiliar with electronics in order to build the sensing network (a plug and play solution).

\hfill bdm

\hfill April 25, 2018

\subsection{Definitions}
Internet of Things: IOT
% needed in second column of first page if using \IEEEpubid
%\IEEEpubidadjcol
\section{Motivation}

\section{Architecture}

\section{Communication}
LoRa is a low power, long range telecommunication technology that is ideal for the type of energy stringent wireless networks being considered. The physical layer of LoRa (Long Range) is a proprietary spread spectrum technique derived from Chirp Spread Spectrum (CSS) and is owned by Semtech~\cite{borLORAforIOT}. The standard MAC layer protocol is LoRaWAN and is an open standard being developed by the LoRa Alliance~\cite{ouluPetajajarvi}. As this protocol is open source, ad hoc networks can be set up without having to wait for a network provider. The transmission distance, energy usage and data rate are all dependent on five configurable parameters. The Semtech data sheets~\cite{lora:sx1272/73} specify how these parameters affect power usage.  Semtech also offer a calculator~\cite{loraCALCULATOR} that may be used to test individual parameter sets.



\section{Energy Harvesting}
\input{hardware.tex}

\section{Data Handling}
The data handling is organised into 3 sections: The first being a mongodb datastore for all the incoming data from the flowers, the second being the dashboard for visualising the data available in the datastore and the last being a central server that communicates with a lora-module via serialport, parsing and saving the lora packets and also acting as a web-server to the dashboard.

\subsection{MongoDB Datastore}

\subsection{Dashboard}

\subsection{Central Server}
%TODO: Include an image of the entire pipeline

\section{Mechanical Design}
\input{mech.tex}

\section{Conclusion and Future Work}


% Can use something like this to put references on a page
% by themselves when using endfloat and the captionsoff option.
\ifCLASSOPTIONcaptionsoff
  \newpage
\fi



% trigger a \newpage just before the given reference
% number - used to balance the columns on the last page
% adjust value as needed - may need to be readjusted if
% the document is modified later
%\IEEEtriggeratref{8}
% The "triggered" command can be changed if desired:
%\IEEEtriggercmd{\enlargethispage{-5in}}

% references section

% can use a bibliography generated by BibTeX as a .bbl file
% BibTeX documentation can be easily obtained at:
% http://mirror.ctan.org/biblio/bibtex/contrib/doc/
% The IEEEtran BibTeX style support page is at:
% http://www.michaelshell.org/tex/ieeetran/bibtex/
%\bibliographystyle{IEEEtran}
% argument is your BibTeX string definitions and bibliography database(s)
%\bibliography{IEEEabrv,../bib/paper}
%
% <OR> manually copy in the resultant .bbl file
% set second argument of \begin to the number of references
% (used to reserve space for the reference number labels box)
\printbibliography



% that's all folks
\end{document}


